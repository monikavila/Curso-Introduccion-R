\begin{frame}{Hoja de Referencia: Vectores}

Se puede crear un vector de distintas formas:

\begin{enumerate}
\tightlist
\item
  \texttt{nombrevector <- c( , )}
\item
  \texttt{nombrevector <- limiteinferor:limitesuperior}
\item
  Usar generacion aleatoria (lo veremos mas tarde).
\item
  Para ver si es un vector usar la funcion:
  \texttt{is.vector(vectorint)}
\end{enumerate}
\end{frame}

\begin{frame}[fragile]{Hoja de referencia: Ejemplos}

\begin{block}{Create vectors}

Creamos un vector sin asignarlo.

\begin{Shaded}
\begin{Highlighting}[]
\DecValTok{1}\SpecialCharTok{:}\DecValTok{10}
\end{Highlighting}
\end{Shaded}

\begin{verbatim}
##  [1]  1  2  3  4  5  6  7  8  9 10
\end{verbatim}

Creamos un vector asignándolo.

\begin{Shaded}
\begin{Highlighting}[]
\NormalTok{vectorint}\OtherTok{\textless{}{-}}\DecValTok{1}\SpecialCharTok{:}\DecValTok{10}

\NormalTok{vectorint}
\end{Highlighting}
\end{Shaded}

\begin{verbatim}
##  [1]  1  2  3  4  5  6  7  8  9 10
\end{verbatim}
\end{block}
\end{frame}

\begin{frame}[fragile]{Hoja de referencia:  Ejemplos}
\begin{block}{Como saber que tipo de vector creamos?}
\begin{Shaded}
\begin{Highlighting}[]
\FunctionTok{is.vector}\NormalTok{(vectorint)}
\end{Highlighting}
\end{Shaded}

\begin{verbatim}
## [1] TRUE
\end{verbatim}

\begin{Shaded}
\begin{Highlighting}[]
\FunctionTok{class}\NormalTok{(vectorint)}
\end{Highlighting}
\end{Shaded}

\begin{verbatim}
## [1] "integer"
\end{verbatim}
\end{block}
\end{frame}


